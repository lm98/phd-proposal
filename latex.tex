\documentclass[12pt]{article}
	\usepackage{mgates-letter}
	\definecolor{dark_blue} {rgb}{0., 0., 0.65}
	
	\usepackage{textcomp}
	\usepackage{mathrsfs}  % mathscr font
	\usepackage{boxedminipage}
	\usepackage{rotating}
	\usepackage{svg}
	%\usepackage{natbib}
	\usepackage[colorlinks, filecolor=dark_blue, urlcolor=dark_blue, linkcolor=black, citecolor=black]{hyperref}
\begin{document}

\title{Towards democratization and resource awareness in dynamic swarm behaviors consigli sul titolo? sul capitolo 2}
\author{Leonardo Micelli}
\date{\today}
\maketitle

\noindent


% ----------------------------------------
\newpage
\setcounter{tocdepth}{2}

% set these after the TOC
\setlength{\parindent}{0em}
\setlength{\parskip}{1em}

% ----------------------------------------
\section{State of the Art}
\paragraph{\textbf{Collective Adaptive Systems and Macroprogramming}} Collective Adaptive Systems~\cite{ferscha2015collective} are large scale distributed systems composed of multiple autonomous, 
interacting entities that operate in open-ended environments. While each entity within the CAS (i.e. sensor, robot, software agent) operates autonomously with its own local context and objective, their interactions
produce \textbf{emergent global behaviors}. Decision-making in CAS is decentralized, and global functionality emerges from local interaction. Examples of CAS can be found in \textit{Smart Cities}, \textit{WSNs}, \textit{Digital Twins} and \textit{Swarm Robotics}.
CAS are tipically heterogeneous and highly dynamic: entities can fail, join and leave at any time, introducing key challenges ranging from scalability and resource constraints to resilience and adaptability in
unpredictable environments. Novel approaches to the engineering of CAS propose to shift the focus from the individual entity to the collective as a whole \textit{(macroprogramming~\cite{10.1145/3579353})}, to manage increasing complexity while
improving maintainability. The main idea is to capture the macroscopic behavior of a collective system through a single program, abstracting away the complexities of individual components.

\paragraph{\textbf{Swarm Robotics}} Swarm robotics is an approach to collective robotics that draws inspiration from the self-organized behavior of social animals ~\cite{brambilla2013swarm}. The goal is to design robust, flexible and scalable 
collective behaviors for large number of robots through simple rules and local interactions. There are two main methods for designing such systems: \textit{automatic design}, which employs methods derived from
evolutionary robotics and multi-robot reinforcement learning, and \textit{behavior-based design}, which involves the design and implementation of algorithms that can be executed by the robots. This design approach is often realized
in a bottom-up fashion, starting from the individual behavior of the robots. Conversely, the top-down approach is based on the idea of expressing the desired behavior of the swarm by expressing a set of instruction at the collective level.
Another important distinction is between \textit{centralized} and \textit{decentralized} approaches. Centralized approaches rely on a central entity that coordinates the behavior of the swarm, while decentralized approaches rely on local interactions between the robots to achieve the desired behavior.
Decentralized approaches, in particular, offer several interesting properties such as robustness, fault tolerance and scalability, while centralized approaches are often more efficient and easier to implement.
The field of swarm robotics has found several applications in various domains, such as \textit{(i)} foraging~\cite{talamali2020sophisticated}, \textit{(ii)} surveillance~\cite{saska2016swarm}, \textit{(iii)} exploration~\cite{huang2019exploration}, \textit{(iv)} transportation and logistics~\cite{zhang2015swarm}.

% Spiegare meglio le proprietà interessanti di AC in ottica CAS
% Spiegare qui le proprietà e i bloccchi self stabilizing con esempi.
% Spiegare che aggregate assume solo la comunicazione con con neighbors e limitate sensing capabilities
\paragraph{\textbf{Aggregate Computing and Field Calculus}} Aggregate Computing (AC)~\cite{beal2016aggregate} is a macroprogramming approach that aims to ease the engineering of CAS by shifting the
focus from the individual device perspective to large aggregations of devices.
In the AC context, a system  is modelled as a set of logical computing nodes (or "devices")
where each node is equipped with sensors and actuators, and is connected with other nodes
according to some neighbouring relationships. This abstract logical model does not prescribe
particular technological solutions; instead, it uses minimal assumptions on the capabilities
of devices. The execution model is based on repeatedly executing computational rounds consisting of
\textit{(i) sense} (gather local context and messages from neighbors), \textit{(ii) compute} (evaluate the aggregate program, producing an output) and 
\textit{(iii) interact} (broadcast the evaluation result to neighbors).

To reason about the properties of computation in AC, it is possible to use the framework of \textit{event structures}~\cite{nielsen1981petri} following the approach proposed by Audrito et. al.~\cite{audrito2018space},  by which a
\textit{field} $\phi:E \rightarrow V$ is a function that maps each event in an event structure $\epsilon$ to a value, 
which can be used to represent the evolution of a distributed property of the system—e.g.,
the field of temperatures perceived by mobile/stationary sensors spreads in the environment,
or the field of movement directions feeding movement actuators.
In this context, an \textit{Aggregate Program P} applied to an event structure $\epsilon$ induces a \textit{field computation} $f:\phi_{in}\rightarrow\phi_{out}$ which is a function mapping an input field to an ouput field, formed by values at each event.

Based on the field abstraction, the Field Calculus (FC)~\cite{viroli2013calculus} provides a minimal language to express distributed, field-based computations consisting of stateful field evolution over time, neighbor interaction and splitting computation domains.
By combining AC's execution model and program specifications based functional composition of FC operators, it is possible to achieve
adaptiveness and global emergent behavior. 
Viroli et. al.~\cite{viroli2018engineering} demonstrated that, for \textit{fair} event structures, fields $\phi$ are said to \textit{converge}, meaning that for each device in an event structure, $\phi$ eventually assigns a value to it. 
Moreover, some field computations were identified as \textit{self-stabilizing}, i.e., once topology and input data are fixed, computation eventually reaches a final
stable configuration of output values, in spite of transient faults. Some of these high-level operators include:

\begin{itemize}
	\item \textit{Sparse Choice S:} yield a self-stabilising Boolean field which is true in a sparse set of devices located at a mean distance grain.
	\item \textit{Gradient-cast G:} propagate a value from source devices outwards along the gradient of increasing distances from them, transforming the value through an
accumulation function along the way.
	\item \textit{Collect-cast C:} summarise distributed information into sink devices, the values provided by devices around the system, while aggregating information through an accumulation function along
the gradient directed towards the sinks.
\end{itemize}

AC is being actively researched in many areas, such as reinforcement learning~\cite{aguzzi2022towards}, IoT~\cite{beal2015aggregate} and digital twins~\cite{casadei2021digital}.

\paragraph{\textbf{Aggregate Computing Incarnations}} Research on AC has led to the development of several incarnations, each of them tackling various research challenges of AC.
\textit{Scafi}~\cite{casadei2016towards} is one of the most actively researched and maintained implementations of
AC. It is hosted in the Scala language, a powerful and expressive JVM-based language. The main advantage of Scafi is its ability to provide a
more high-level platform to support agile prototyping for research. \textit{Collektive} is a Kotlin-based implementation of AC that provides an extension of FC via the eXchange Calculus (XC) ~\cite{audrito2024exchange}.
It provides an expressive DSL and it is natively multi-platform, enabling AC on a wider range of targets.
\textit{FCCP}~\cite{audrito2024fcpp} is a C++ library that implements FC. It has been designed and developed to bring the AC paradigm to
resource-constrained devices that cannot support the JVM. It does so by providing an extensible C++ library and a performance-oriented simulator that allows
the developer to speed up the development process of aggregate programs. 

\subsection{Coherence with my previous academic experience}
\label{sec:coherence}
This proposed research projects extends and builds upon my previous academic experience, in particular in my last year of Master's degree.
I have been introduced to AC and Scafi during the course of "Pervasive computing". Here, in the context of the final examination project for the course,
I worked alongside three colleagues of mine to develop a Rust-based imlpementation of Aggregate Computing and a Scala 3 port of the Scafi DSL.
I then built upon this project to develop my Master's thesis where I proposed a Rust-based distributed framework to execute distributed AC programs in a network of
heterogeneous devices, with the main goal of democratizing AC by offering a high-level API that can be supported in resource-constrained devices.

% ----------------------------------------
\section{Description of the Project}
\subsection{Motivation}
The analysis from Brambilla~\cite{brambilla2013swarm} highlighted a gap in research on top-down design methods of collective behaviors.
In particular, there is a need for a unified framework that enables expressing at the macro/collective level swarm behaviors that have
been of research interest such as:
\textit{(i)} consensus~\cite{valentini2017achieving}, \textit{(ii)} leader election~\cite{karpov2015leader}, \textit{(iii)} pattern formation~\cite{sahin2002swarm},
\textit{(iv)} team organization~\cite{nouyan2009teamwork} and \textit{(v)} collective planning~\cite{sampedro2016flexible}.

In particular, there is a lack of a unified solution that can express, from a collective perspective, the aforementioned behaviors while granting decentralization, 
resiliency, self-stabilization and guarantee of convergence, especially in low-resource and low-assumption contexts such as
bearing-only formations~\cite{zhao2021bearing}, where data about orientation is gathered from simple sensors and used to form resilient swarm formations.

Recent preliminary work in this direction has been conducted with MacroSwarm~\cite{aguzzi2023macroswarm}, a AC-powered ScaFi extension that exposes an API of collective
behaviors built on top of AC's resilient operators. However, deeper research and further expansion is needed in areas such as dynamic team formation, resource management and adaptive workflow management and planning.
Finally, Scafi's dependency on the JVM united with the variegated scenario regarding AC incarnations highlights the need of a \textit{lingua franca} that enables the definition and deployment
of collective behaviors in heterogeneous and possibly resource-constrained systems.

\subsection{Idea}
% Oiù che parlare di democratizzazione parlare di lingua franca e spostare nelle motivazioni

descrivere in modo chiaro le proprietà che vogliamo in questi sistemi

The idea of the project, exemplified in Figure~\ref{fig:research-project}, is to provide a top-down, unified approach to swarm behavior design, by offering a
set of resilient, functional and composable blocks that will form a general purpose API for swarm robotics. These blocks will guarantee decentralization, self-adaptiveness and resiliency.
Given Aggregate Programs' proven~\cite{viroli2018engineering} resiliency, self-adaptability and self-stabilization properties, an AC platform will serve as the
foundation upon which functional blocks for \textit{(i)} consensus, \textit{(ii)} leader election, \textit{(iii)} pattern formation,
\textit{(iv)} team organization and \textit{(v)} collective planning will be built. This can be done thanks to the possibility of leveraging FC's resilient operators of sparse choice, 
gradient-cast and collect-cast alongside other core FC operators. Moreover, the proposed AC platform will take into account the need for interoperability between devices with different 
computational capabilities to ensure the possibility of real-world deployment.

\begin{figure}
	\centering
	\includegraphics[width=0.7\textwidth]{figures/ResearchProject.png}
	\caption{Overview of the research project}
	 \label{fig:research-project}
\end{figure}

\subsection{Research Goals and Challenges}
\label{sec:challenges}
todo: definire meglio cosa si intende il soggetto della challenge e fare qualche citazione per giustificare il fatto che le research challenge sono interessanti

\subsubsection{Team Management}
CAS are naturally dynamic and continually evolving. One implication of these properties is that entities within a CAS can fail, join and leave over the course of time.
Within a collective computation and AC, the sudden change in the neighborhood of the composing entities can yield to different outputs, and this effect may propagate upwards and
affect the collective result of the computation. In AC-based swarm robotic contexts, where swarm behaviors are mappings between sensing fields to actuation fields, sudden changes
in the actual underlying network of robots can cause the inability of the entire system to reach the desired collective goal. Ensuring that the swarm behavior expressed through the
framework is resilient to sudden topology change is crucial to provide stable behavior. 

\subsubsection{Swarm Formation}
spiegare che il team è una suddivisione prettamente logica (magari goal oriented).
All'interno del team lo swarm può strutturarsi in una organizzazione topologica (formation)

Team formation patterns are one of the key challenges of engineering swarm behaviors. 
Teams can be formed from the swarm following a particular logic, sensor reading or spatial structure. Usually teams 
are associated to a leader (introducing the challenge of \textit{leader election}) which can function as an area of influence in the
local decision of the neighboring nodes. Recent research show the need of advancements in the following categories:
\textit{(i) decentralization}: the formation behavior needs to emerge from local interaction between robots in the swarm, rather than from a centralized entity. 
\textit{(ii) resilience}, meaning the ability of the swarm to maintain formation/decision
in the presence of adversarial events, in particular \textit{message loss, perception position errors} and \textit{node failures}.
\textit{(iii) self-healing}, i.e. the ability of the formation to reconstruct itself and return to a stable structure if disturbances in the structure happen. Recent progress has been made
regarding self-healing structures, in particular MacroSwarm offers a set of structures that have been proven as self-healing. However, deeper research in the topic is needed to generalize self-healing
to all possible swarm formation structures. dire che mancano garanzie forti su tempi di convergenza e risposta ai rumori

\subsubsection{Planning}
cosa si intende con planning e workflow management? Fare esempi
Si potrebbe citare Rise of the Swarm (eventualmente citabile anche nelle motivazioni)

Planning in swarm robotics is the distributed process by which the collective dynamically coordinates their actions to achieve complex, adaptive objectives.
In top-down, macro level approaches, it is possible to devise an algorithm expressing the collective goal of the swarm by combining swarm behaviors with goals represented by boolean conditions.
While this approach shows a step in the right direction, swarms may be deployed in highly mutable environments, where plan adaptivity and dynamic resource allocation based on environmental conditions
are crucial to allow the swarm to reach its final goal. Nodes may distribute themselves across different independent tasks based on local status and condition, or may adapt their current global behavior
based on available resources in order to prevent wear-off or save energy.
This highlights the need for a mechanism to express resource, condition aware plans and dynamic workflows.

\subsubsection{Democratization of Aggregate Computing}
%caratteristiche di democratic AC. Rustfields?
In the context of the proposed project, a \textit{democratic} incarnation of Aggregate Computing \textit{(i)} provides an expressive, general purpose and
composable API that enables the design, development and maintenance of complex CAS (user-oriented democratization), \textit{(ii)} enables the deployment
in the largest portions of systems, even highly resource-constrained ones (device-oriented democratization). Current AC incarnations offer different properties
regarding this matter, in particular Scafi is highly expressive but with limited device coverage, while FCPP can be deployed in limited resources environments
at the cost of ergonomics. Alongside this spectrum, there are incarnations that have the capability to cover more devices while maintaining expressiveness, such as the Kotlin-based
Collektive framework and other solutions are being explored with Rust-based incarnations as mentioned in Section~\ref{sec:coherence}. It will be crucial to the project
to ensure that the AC platform exploited by the framework provides the aforementioned properties.

\subsection{Reference Scenarios}
\label{sec:scenarios}
% Motivare un po' per ogni scenario perché swarms e le motivazioni sono utili in questi scenarri
% Fare qualche esempio su swarm robotics ma anche generali su cas come smart cities

%\paragraph{Disaster Response} One of the most promising areas is disaster response and search \& rescue. Swarms of inexpensive robots can rapidly explore a disaster site, 
%locating survivors or mapping damage much faster than a single robot or human team. 
%Aerial drone swarms could coordinate to cover different sectors of a search area, communicate sightings of survivors, and collaboratively haul lightweight relief supplies. Their distributed nature means even if some drones crash or lose signal, the rest can complete the mission - a critical advantage in hazardous environments.

\paragraph{Environmental Monitoring and Mapping}
In this scenario, a swarm of small, ground-based or aerial robots could collaboratively explore, map, and monitor a defined natural environment, such as a forest, agricultural field, or coastal area. 
Their primary objectives could include identifying environmental anomalies (e.g., detecting pollution, monitoring vegetation health), creating a detailed real-time map of the area, 
and dynamically adapting their collective movement based on environmental cues or unexpected events.

\paragraph{Digital Twins of Swarms}
Swarm Digital Twins represent virtual replicas of physical swarm robotic systems, capturing their collective behaviors and interactions in real-time. 
In this scenario, a digital twin of a robot swarm deployed for environmental monitoring continuously integrates real-time sensor data and operational states of each physical robot. 
Leveraging this digital counterpart, operators can anticipate emergent behaviors, optimize swarm performance, and preemptively detect faults or potential disruptions before they manifest in the real system. 
Furthermore, digital twins provide a safe platform for testing and validating new swarm algorithms and coordination strategies prior to physical deployment, significantly reducing experimentation costs and risks. 
By merging% Spiegare meglio le proprietà interessanti di AC in ottica CAS
 real-time physical insights with advanced predictive analytics, swarm digital twins enable more efficient, reliable, and adaptable swarm robotic systems, bridging the gap between theoretical models and practical applications.

%Aggiungere scenari CAS generali come crowd management??
% ----------------------------------------
\section{Expected Results}
As a form of result of the proposed project, it is expected to bring \textit{(i)} scientific contributions to CAS engineering, in particular in swarm robotics and collective behavior through AC,
\textit{(iii)} to realize an actionable top-down design framework based on AC to address the open challenges in Section~\ref{sec:challenges}, \textit{(iii)} a real-world validation possibly based on the reference scenarios in Section~\ref{sec:scenarios}.

The future vision for the project is to enable the building of complex, resilient and dynamic swarm behaviors and deploy them in real-world scenarios, some of which could be \textit{(i)} precision agriculture,
\textit{(ii)} environmental monitoring, \textit{(iii)} logistics and warehousing, \textit{(iv)} disaster response and search, 
\textit{(v)} ubiquitous computing with Swarm Digital Twins.

menzionare anche il simulated environment
% ----------------------------------------
\section{Lead-time for implementation}
%Simulazioni con argos alchemist ecc
\begin{figure}
	\includegraphics[width=\linewidth]{figures/timeline.png}
	\caption{Hypothetical three year timeline for the project.}
	\label{fig:timeline}
\end{figure}

The Figure~\ref{fig:timeline} shows the hypothetical three-year timeline for the project, subdivided into quarters.

\paragraph{First year.} In the first year will be crucial a deep dive into the research of the state of the art in aggregate computing platforms to devise the design of the Aggregate Computing Platform.
In parallel, deep research of current FC operators and their properties will lay the foundation of the basic composable building blocks for swarm behavior, such as flocking, pattern formation, consensus and leader election.
Due to the overlap of these two tasks, it is expected for the latter to carry over the first half of the second year, to allow enough time for the AC platform to be completed by the end of the first year.

\paragraph{Second year.} Within the second year, it is expected to shift focus from the basic building blocks of swarm behavior to the extended library, that will tackle some of the most relevant challenges expressed in Section~\ref{sec:challenges}.
There will also be an increase in prototyping activities, which will first start in a simulated environment.
Towards the end of the year, it is expected to start leverage swarm behaviors to tackle a reference scenario in the real-world.

\paragraph{Third year.} In the third and final year, the focus will predominantly be to finalize the library of resilient swarm behaviors, while also converging to the end of the real-world testbed in the referenced scenario.

% ----------------------------------------
\section{Proposed criteria to be used to assess the findings obtained}

\paragraph{Qualitative.}
The project will be assessed qualitatively by examining the expressiveness, usability, and resilience of the proposed framework in real-world and simulated deployments.  
Key questions include:  
(i) Expressiveness \& composability: possibility to describe a broad spectrum of swarm behaviours concisely. 
(ii) Robustness \& adaptability in the field: how the swarm copes with adversarial events and whether it self‑stabilises to the intended global behaviour.  
(iii) “Democratisation:” Success is achieved if the same aggregate program can run unmodified on heterogeneous hardware classes that include low resource devices.


\paragraph{Quantitative.}
Experiments need to be rigorously conducted and documented from the beginning of the simulated prototype phase until the end of the reference scenario implementation in the real world.
Detailed quantitative metrics for success need to be expressed while documenting such experiments. 
The metrics gathered will be compared with state-of-the-art swarm robotics frameworks to evaluate performance relative to optimal solutions.
As field-testing requires specialized hardware, efforts will be made to secure partnerships for international research stays during the PhD program, facilitating access to necessary equipment for project testing.

\paragraph{Scientific Contributions.}
During the course of the program, it is expected to contribute to the scientific knowledge of CAS engineering and swarm robotics. 
In particular, it is expected to submit papers annually to conferences that are relevant to the field of CAS (such as ACSOS \footnote{\url{https://acsos.github.io/}}) and robotics (such as ICRA \footnote{\url{https://2025.ieee-icra.org/}}).
TAAS e SwarmIntelligence come journals 
\clearpage

%-----------------------------------------
\renewcommand{\refname}{References}

\bibliographystyle{plain}
\bibliography{latex}

\end{document}
