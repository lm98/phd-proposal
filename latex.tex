\documentclass[12pt]{article}
	\usepackage{mgates-letter}
	\definecolor{dark_blue} {rgb}{0., 0., 0.65}
	
	\usepackage{textcomp}
	\usepackage{mathrsfs}  % mathscr font
	\usepackage{boxedminipage}
	\usepackage{rotating}
	%\usepackage{natbib}
	\usepackage[colorlinks, filecolor=dark_blue, urlcolor=dark_blue, linkcolor=black, citecolor=black]{hyperref}
\begin{document}

\title{Towards a democratized landscape for the development of Collective Adaptive Systems}
\author{Leonardo Micelli}
\date{\today}
\maketitle

\noindent


% ----------------------------------------
\newpage
\setcounter{tocdepth}{2}
\tableofcontents

% set these after the TOC
\setlength{\parindent}{0em}
\setlength{\parskip}{1em}


% ----------------------------------------
\newpage
\section{Context and Motivation}
(max. 1000 characters)

In the rising context of ubiquitous computing \cite{weiser1991computer},
Collective Adaptive Systems (CAS) have emerged as a fundamental paradigm to
address the challenges of developing large-scale, dynamic and complex distributed
systems where multiple, heterogeneous and often resource-constrained entities
interact and coordinate to achieve a global behavior.

Aggregate Computing (AC) is a programming paradigm, originated from Field Calculus \cite{viroli2013calculus},
that aims to provide a unified framework for developing CAS by abstracting away the complexities
of distributed systems and instead focusing on the collective behavior of the system.

Many state-of-the-art implementations of AC have been proposed, each of them exploiting different platforms
and technologies. While each of these offers unique sets of advantages, there is also a lack of a unified, interoperable 
and democratized approach to AC where entities with different capabilities and resources can seamlessly interact and collaborate.

% ----------------------------------------
\newpage
\section{State of the Art}
(max. 2500 characters)

\textbf{Collective Adaptive Systems} Collective Adaptive Systems are composed of multiple autonomous, interacting entities that operate in open-ended environments. 
They are capable of self-organization, self-adaptation, and emergent coordination—properties crucial for systems that must continue to function under continuous change and uncertainty.
In CAS, global behavior arises not from centralized planning but from local interactions among components, much like natural collectives such as ant colonies or bird flocks.

\textbf{Macroprogramming}\cite{10.1145/3579353} is a paradigm that expresses the macroscopic behavior of a collective system using a single program, with the aim of
capturing the global behavior of the system while abstracting away the complexities of individual components.

\textbf{Aggregate Computing and Field Calculus}. AC\cite{beal2016aggregate} is a programming approach that aims to ease the engineering of CAS by shifting the
focus from the individual device perspective to large aggregations of devices. It does so by exploiting the concepts of computational fields and FC.
Within the FC, a computational field is a function mapping every computational device in a network, represented by a dynamic and reflexive neighboring
relationship between devices, to a computational object. The main goal is to express the aggregate system behavior
through a functional composition of fundamental operators that manipulate (evolve, combine, restrict)
computational fields. A key concept of Field Calculus is that these aggregate-level specifications can
also be interpreted as a local set of rules that define the iterative asynchronous
execution of computation rounds.

\textbf{Scafi}\cite{casadei2016towards} is one of the most actively researched and maintained implementations of
AC. It is hosted in the Scala language, a powerful and expressive JVM-based
language. The main advantage of Scafi is its ability to provide a
more high-level platform to support agile prototyping for research.

\textbf{Collektive} is a Kotlin-based implementation of AC that provides an extension of FC via the eXchange Calculus (XC) \cite{audrito2024exchange}.
It provides an expressive DSL and it is natively multi-platform, enabling AC on a wider range of targets.

\textbf{FCCP}\cite{audrito2024fcpp} is a C++ library that implements FC. It has been designed and developed to bring the AC paradigm to
resource-constrained devices that cannot support the JVM. It does so by providing an extensible C++ library and a performance-oriented simulator that allows
the developer to speed up the development process of aggregate programs.

% ----------------------------------------
\newpage
\section{Project Proposal}
max 7500 characters

% ----------------------------------------
\newpage
\section{Expected Results}
max. 3500 characters

% ----------------------------------------
\newpage
\section{Timeline}
max. 2000 characters

% ----------------------------------------
\newpage
\section{Verification Criteria}
max. 3500 characters

\clearpage

%-----------------------------------------
\renewcommand{\refname}{References}

\bibliographystyle{plain}
\bibliography{latex}

\end{document}
